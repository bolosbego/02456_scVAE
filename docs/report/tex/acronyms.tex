%% HOW TO CREATE AN ACRONYM
% \newacronym{1}{2}{3}
% 1: the LaTeX name of the acronym, you use this when you mention the acronym in the text with \gls{}, \Gls{} or \glspl{}
% 2: the acronym that is displayed
% 3: the full text what the acronym stands for, that is printed when \gls{} is used for the first time for this acronym
% example: \newacronym{ppd}{PPD}{Pharmaceutical Process Development}
% Usage in code then is: \gls{ppd} 
% \glspl{ppd} would give you an s after the acronym+text. pl = plural
% \Gls{} forces the first letter to be uppercase

% TO PRINT ACRONYM LIST
% use this command in main.tex:
% \printglossary[type=\acronymtype ,title=List of abbreviations]
% and make sure command \makeglossaries has been called before

\newacronym{ffnn}{FFNN}{feed-forward neural network}
\newacronym{vae}{VAE}{variational auto-encoder}
\newacronym{scvae}{scVAE}{single-cell variational auto-encoder}
\newacronym{scrna-seq}{scRNA-seq}{single-cell RNA sequencing}
\newacronym{rna-seq}{RNA-seq}{RNA sequencing}
\newacronym{lr}{LR}{latent representation}
\newacronym{pca}{PCA}{principal component analysis}
\newacronym{tsne}{tSNE}{t-distributed stochastic neighbor embedding}
\newacronym{elbo}{ELBO}{evidence lower bound}
\newacronym{ngs}{NGS}{next-generation sequencing}
\newacronym{pbmc}{PBMC}{peripheral blood mononuclear cells}
\newacronym{gmvae}{GMVAE}{gaussian mixture variational auto-encoder}